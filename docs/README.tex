% Created 2022-04-16 sáb 21:20
% Intended LaTeX compiler: pdflatex
\documentclass[11pt]{article}
\usepackage[utf8]{inputenc}
\usepackage[T1]{fontenc}
\usepackage{graphicx}
\usepackage{longtable}
\usepackage{wrapfig}
\usepackage{rotating}
\usepackage[normalem]{ulem}
\usepackage{amsmath}
\usepackage{amssymb}
\usepackage{capt-of}
\usepackage{hyperref}

\usepackage{amsthm}
\theoremstyle{definition}
\newtheorem{teo}{Teorema}[section]
\theoremstyle{definition}
\newtheorem{defi}{Definicao}[section]
\theoremstyle{remark}
\newtheorem{obs}{Observação}[section]
\theoremstyle{remark}
\newtheorem{lema}{Lema}[section]
\theoremstyle{remark}
\newtheorem{prop}{Propriedade}[section]
\theoremstyle{remark}
\newtheorem{coro}{Corolario}[section]
\theoremstyle{definition}
\newtheorem{prep}{Preposição}[section]
\author{Ieremies Romero, Gian Franco, Gabriel Branco}
\date{\today}
\title{Atividade 2 - MO824}
\hypersetup{
 pdfauthor={Ieremies Romero, Gian Franco, Gabriel Branco},
 pdftitle={Atividade 2 - MO824},
 pdfkeywords={},
 pdfsubject={},
 pdfcreator={Emacs 27.2 (Org mode 9.6)}, 
 pdflang={Portuguese}}
\usepackage{biblatex}
\addbibresource{~/arq/bib.bib}
\begin{document}

\maketitle


\section*{Modelo}
\label{sec:org5ad2af7}

Usamos o modelo já conhecido para o TSP convencional como base

\begin{align*}
\min \sum c_{e} x_{e}& \\
\sum \limits_{e \in \delta(v)} x_{e} &= 2 v \in V \\
\sum \limits_{e \in \delta(S)} x_{e} &\leq |S| -1 S \subset V
\end{align*}

No nosso caso, temos que resolver dois TSP's mas que as soluções possuam \(k\) arestas em comum.

No nosso modelo \(x_{e}^{1}\) indica que usamos a aresta \(e\) para o tuor \(1\) o respectivo para o tuor 2 e \(D_{e}\) indica se a aresta está duplicada.

Nossa função objetivo pode ser a soma dos custos dos dois tuors, ou seja
\[  \min \sum \limits_{e \in E} \sum \limits_{i \in \{1,2\}} c_{e} x_{e}^{i}. \]

Repetimos as restrições do TSP para cada um dos tuors.
\begin{align*}
\sum \limits_{e \in \delta(v)} x_{e}^{i} &= 2 v \in V \ \forall i \in \{1,2\} \\
\sum \limits_{e \in \delta(S)} x_{e}^{i} &\leq |S| -1 \ \forall S \subset V \ \forall i \in \{1,2\}
\end{align*}

É importante notar que a segunda equação dá origem a quantidade exponencial de restrições de eliminação de subtuor. No nosso código, podemos circundar esse problema adicionando as restrições conforme se faz necessário. Assim, quando o modelo termina com um certo conjunto de restrições, podemos conferir, por meio de uma busca de profundidade, se é uma solução viável considerando a restrição de subtuor. Caso não seja, adicionamos as restrições de subtuor que evitam essa solução. Fazemos isso até encontrarmos uma solução viável.

Por fim, adicionamos as restrições que exigem a quantidade de arestas compartilhadas.
\begin{align*}
x_{e}^{i} &\geq D_{e} \ \forall e \in E \ \forall i \in \{1,2\} \\
\sum \limits_{e \in E} D_{e} &\geq k
\end{align*}

Assim, nosso modelo final é
\begin{align*}
\min \sum \limits_{e \in E} \sum \limits_{i \in \{1,2\}} c_{e} x_{e}^{i}& \\
\sum \limits_{e \in \delta(v)} x_{e}^{i} &= 2 \ \forall v \in V \ \forall i \in \{1,2\} \\
\sum \limits_{e \in \delta(S)} x_{e}^{i} &\leq |S| -1 \ \forall S \subset V \forall i \in \{1,2\} \\
x_{e}^{i} &\geq D_{e} \ \forall e \in E \ \forall i \in \{1,2\} \\
\sum \limits_{e \in E} D_{e} &\geq k
\end{align*}

\section*{Geração de instâncias}
\label{sec:org682535d}
Para testar nosso modelo, utilizamos o arquivo de coordenadas disponibilizado pelo professor para calcular nossos custos de arestas. Assim, para instâncias de 100 cidades, utilizamos as 100 primeiras linhas do arquivo.

Durante os testes, modificamos a quantidade de cidades (\(100\), \(150\), \(200\) e \(250\)) e o valor de \(k\) (zero, metade da quantidade de cidades e a quantidade de cidades).
\section*{Resultados}
\label{sec:org130df22}

Realizamos os testes em um computador equipado de um processador i5 de oitava geração, com 4 cores e 8 threads a 1.6ghz (max boost 3.2) e 8gb de ram, sem swap, com sistema operacional Linux 64bits. 


\begin{table}[htbp]
\caption{Métricas do modelo para as instâncias citadas no formato custo, gap e tempo de execução.}
\centering
\begin{tabular}{llll}
 & \(k=0\) & \(k=\frac{v}{2}\) & \(k = v\)\\
\hline
\(v = 100\) & (1630, 0\%, 12.46) & (2102, 0\%, 51.23) & (3463, 0\%, 18.29)\\
\(v = 150\) & (1966, 0\%, 79.01) & (2748, 0\%, 565.42) & (4780, 0\%, 36.89)\\
\(v = 200\) & (2308, 0\%, 208.72) & (17763, 80.9\%,2048) & (6003, 0\%, 112.67)\\
\(v = 250\) & (2916, 11.1\%, 1113) & (7525, 0\%, ) & (6999, 0\%, 596.74)\\
\end{tabular}
\end{table}


É importante ressaltar que a instância com \(v=250\) e \(k=0\) foi finalizada pelo sistema operacional por falta de memória ram. O resultado reportado aqui é da última atualização fornecida pelo Gurobi.

\section*{Análise}
\label{sec:org7143c1c}
Observando as métricas obtidas, vemos os piores tempos de execução são encontrados quando \(k = \frac{v}{2}\), com as instâncias maiores demorando consideravelmente a terminar, muitas vezes sem achar uma solução ótima. Além disso, apesar não diretamente mostrado na tabela, observamos que o tempo gasto adicionando restrições de forma "lazy" é consideravelmente pequeno em comparação com o tempo total.
\section*{Fontes}
\label{sec:org013517e}
\url{https://colab.research.google.com/github/Gurobi/modeling-examples/blob/master/traveling\_salesman/tsp\_gcl.ipynb}
\end{document}
