% Created 2022-04-11 seg 22:43
% Intended LaTeX compiler: pdflatex
\documentclass[11pt]{article}
\usepackage[utf8]{inputenc}
\usepackage[T1]{fontenc}
\usepackage{graphicx}
\usepackage{longtable}
\usepackage{wrapfig}
\usepackage{rotating}
\usepackage[normalem]{ulem}
\usepackage{amsmath}
\usepackage{amssymb}
\usepackage{capt-of}
\usepackage{hyperref}

\usepackage{amsthm}
\theoremstyle{definition}
\newtheorem{teo}{Teorema}[section]
\theoremstyle{definition}
\newtheorem{defi}{Definicao}[section]
\theoremstyle{remark}
\newtheorem{obs}{Observação}[section]
\theoremstyle{remark}
\newtheorem{lema}{Lema}[section]
\theoremstyle{remark}
\newtheorem{prop}{Propriedade}[section]
\theoremstyle{remark}
\newtheorem{coro}{Corolario}[section]
\theoremstyle{definition}
\newtheorem{prep}{Preposição}[section]
\author{Ieremies Vieira da Fonseca Romero}
\date{\today}
\title{}
\hypersetup{
 pdfauthor={Ieremies Vieira da Fonseca Romero},
 pdftitle={},
 pdfkeywords={},
 pdfsubject={},
 pdfcreator={Emacs 27.2 (Org mode 9.6)}, 
 pdflang={Portuguese}}
\usepackage{biblatex}
\addbibresource{~/arq/bib.bib}
\begin{document}

\tableofcontents

\section{Modelo}
\label{sec:org51a1b3e}

Usamos o modelo já conhecido para o TSP convencional como base

\begin{align*}
\min \sum c_{e} x_{e}& \\
\sum \limits_{e \in \delta(v)} x_{e} &= 2 v \in V \\
\sum \limits_{e \in \delta(S)} x_{e} &\leq |S| -1 S \subset V
\end{align*}

No nosso caso, temos que resolver dois TSP's mas que as soluções possuam \(k\) arestas em comum.

No nosso modelo \(x_{e}^{1}\) indica que usamos a aresta \(e\) para o tuor \(1\) o respectivo para o tuor 2 e \(D_{e}\) indica se a aresta está duplicada.

Nossa função objetivo pode ser a soma dos custos dos dois tuors, ou seja
\[  \min \sum \limits_{e \in E} \sum \limits_{i \in \{1,2\}} c_{e} x_{e}^{i}. \]

Repetimos as restrições do TSP para cada um dos tuors.
\begin{align*}
\sum \limits_{e \in \delta(v)} x_{e}^{i} &= 2 v \in V \ \forall i \in \{1,2\} \\
\sum \limits_{e \in \delta(S)} x_{e}^{i} &\leq |S| -1 \ \forall S \subset V \ \forall i \in \{1,2\}
\end{align*}

É importante notar que a segunda equação dá origem a quantidade exponencial de restrições de eliminação de subtuor. No nosso código, podemos circundar esse problema adicionando as restrições conforme se faz necessário. Assim, quando o modelo termina com um certo conjunto de restrições, podemos conferir, por meio de uma busca de profundidade, se é uma solução viável considerando a restrição de subtuor. Caso não seja, adicionamos as restrições de subtuor que evitam essa solução. Fazemos isso até encontrarmos uma solução viável.

Por fim, adicionamos as restrições que exigem a quantidade de arestas compartilhadas.
\[ x_{e}^{i} &\geq D_{e} \ \forall e \in E \ \forall i \in \{1,2\} \]

Assim, nosso modelo final é
\begin{align*}
\min \sum \limits_{e \in E} \sum \limits_{i \in \{1,2\}} c_{e} x_{e}^{i}& \\
\sum \limits_{e \in \delta(v)} x_{e}^{i} &= 2 \ \forall v \in V \ \forall i \in \{1,2\} \\
\sum \limits_{e \in \delta(S)} x_{e}^{i} &\leq |S| -1 \ \forall S \subset V \forall i \in \{1,2\} \\
x_{e}^{i} &\geq D_{e} \ \forall e \in E \ \forall i \in \{1,2\}
\end{align*}
\section{Fontes}
\label{sec:orga22c22d}
\url{https://colab.research.google.com/github/Gurobi/modeling-examples/blob/master/traveling\_salesman/tsp\_gcl.ipynb}
\end{document}
